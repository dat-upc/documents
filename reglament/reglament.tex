\documentclass[a4paper,12pt]{article}

\usepackage[catalan]{babel}
\usepackage{parskip}
\usepackage{graphicx}
\usepackage[shortlabels]{enumitem}
\usepackage[hidelinks]{hyperref}

% Canviar la font del document
\usepackage{fontspec}
\setmainfont{Libertinus Serif}

% Evitar que es faci reset al comptador dels articles
\usepackage{chngcntr}
\counterwithout{subsubsection}{subsection}

% Canviar el prefix de l'índex
\usepackage{tocloft}
\renewcommand*{\cftsecpresnum}{Capítol }
\setlength\cftsecnumwidth{5.5em}
\renewcommand*{\cftsubsecpresnum}{Secció }
\setlength\cftsubsecnumwidth{4em}
\renewcommand*{\cftsubsubsecpresnum}{Article }
\setlength\cftsubsubsecnumwidth{5em}

% Canviar el nom de les seccions, subseccions, etc.
\usepackage{titlesec}
\renewcommand*{\thesection}{\Roman{section}.}
\renewcommand*{\thesubsection}{\arabic{subsection}.}
\renewcommand*{\thesubsubsection}{\arabic{subsubsection}.}
\titleformat{\section}[block]
{\normalfont\Large\bfseries}{Capítol~\thesection}{1em}{}
\titleformat{\subsection}
{\normalfont\large\bfseries}{Secció~\thesubsection}{1em}{}
\titleformat{\subsubsection}
{\normalfont\normalsize\bfseries}{Article~\thesubsubsection}{1em}{}

% Títol del document
\title{Reglament de la Delegació d'Alumnes de Telecomunicacions}
\makeatletter

\begin{document}
% Portada
\thispagestyle{empty}
\begin{center}
\includegraphics{Dat_logo_10x6.png}
\vspace{3em}

\Huge
\@title
\vspace{1em}

\large
\today
\end{center}
\normalsize
\newpage

\thispagestyle{empty}
\tableofcontents
\newpage

\section{Disposicions generals}
\subsubsection{Naturalesa}
D’acord amb l’Article 98.1 dels Estatuts de la UPC, la Delegació d’Estudiants és l’òrgan de coordinació dels representants dels estudiants de grau i màster universitari dins l’àmbit de cada centre docent.

Aquest òrgan es defineix com la Delegació d’Alumnes de Telecomunicacions. A partir d’aquest moment es definirà i s’anomenarà DAT o Delegació en aquest document, així com en la seva identitat física i digital.

\subsubsection{Composició i mandat}
La composició de la Delegació d’estudiants és la que s’estableix a l'article 35 del Reglament d’organització i funcionament de l’ETSETB. A més a més, formaran part de la Delegació els claustrals que són estudiants de l’Escola.

\subsubsection{Funcions}
La Delegació d’estudiants té les funcions que li atorguen els Estatuts de la UPC, les Directrius per a l’elaboració dels Reglaments d’organització i funcionament dels centres docents, i el Reglament d’organització i funcionament de l’ETSETB.

La Delegació d’estudiants també té les següents funcions:
\begin{enumerate}[a)]
	\item Elaborar anualment les línies estratègiques i objectius de la Delegació d’estudiants.
	\item Gestionar i fer la liquidació del pressupost que té assignat la Delegació.
	\item Informar als estudiants de grau i màster sobre la tasca que duu a terme la Delegació, així com d’altres aspectes que puguin ser d’interès pels estudiants.
	\item Potenciar i canalitzar la participació dels estudiants en els òrgans de govern de la Universitat i de l’Escola.
	\item Contribuir a l’assessorament dels estudiants de l’Escola i vetllar pels seus drets.
\end{enumerate}

\section{Drets i deures}
\subsubsection{Drets dels membres de la delegació}
A més dels drets que es reconeixen al Reial Decret 1791/2010, de 30 de desembre, pel qual s’aprova l’Estatut de l’estudiant universitari, als Estatuts de la UPC i la resta de normativa que sigui d’aplicació els membres de la delegació tenen els següents drets:

\begin{enumerate}[a)]
	\item Expressar amb llibertat les manifestacions que realitzin en l’exercici de les seves funcions.
	\item Participar a les sessions dels òrgans als quals es pertanyi amb veu i vot.
	\item Ser elector i elegible per a càrrecs de gestió i representació de la Delegació d’estudiants o per formar part de les seves comissions, si escau.
	\item Accedir a les instal·lacions, materials i documents de la Delegació en els termes que s’estableixin
	\item No ser perjudicat per exercir la representació dels estudiants.
	\item Fer ús de la paraula i intervenir d’acord amb els deures de convivència.
\end{enumerate}

\subsubsection{Deures dels membres de la delegació}
A més dels deures que es detallen al Reial Decret 1791/2010, de 30 de desembre, pel qual s’aprova l’Estatut de l’estudiant universitari, als Estatuts de la UPC i la resta de normativa que sigui d’aplicació els membres de la delegació tenen els següents deures:

\begin{enumerate}[a)]
	\item Complir amb aquest Reglament i els acords que s’adoptin.
	\item Assistir a les sessions del Ple i als actes que requereixen la seva presència, així com a la resta d’òrgans als quals es pertanyi.
	\item Transmetre les opinions i peticions col·lectives dels seus representats.
	\item Informar als seus representats de les activitats realitzades en l’exercici del seu càrrec.
	\item Fer un bon ús dels espais, materials, documentació i la resta de mitjans necessaris per a l’exercici de les seves funcions o drets.
	\item Respectar les decisions dels òrgans de la Delegació i no fer pública informació confidencial de la Delegació.
\end{enumerate}

\section{Òrgans col·legiats i unipersonals}
\subsection{Disposicions generals}
\subsubsection{Òrgans col·legiats}
Els òrgans col·legiats de la Delegació d’estudiants són:
\begin{enumerate}[a)]
	\item El Ple.
	\item El Consell de la Delegació.
\end{enumerate}

\subsubsection{Òrgans unipersonals}
Els òrgans unipersonals de la Delegació són:
\begin{enumerate}[a)]
	\item El delegat de centre.
	\item El secretari.
	\item Els sotsdelegats de centre.
	\item El tresorer.
\end{enumerate}

\subsubsection{Disposicions generals sobre els òrgans col·legiats}
\begin{enumerate}[\thesubsubsection1]
	\item L’assistència a les sessions dels òrgans col·legiats té caràcter personal i el vot és intransferible.
	\item La manca d’assistència no excusada a dues sessions consecutives de l’òrgan n’implica el cessament com a membre.
\end{enumerate}

\subsubsection{Convocatòria i sessions dels òrgans col·legiats}
\subsubsection{Adopció d'acords}

\subsection{El Ple}

\subsection{El Consell de la Delegació}

\subsection{El delegat de centre}

\subsection{Els sotdelegats de centre}

\subsection{El secretari}

\subsection{El tresorer}

\section{Aprovació i modificació del Reglament de la Delegació}

\section*{Disposicions addicionals}
\addcontentsline{toc}{section}{Disposicions addicionals}

\subsubsection*{Disposició addicional única. Gènere}
\addcontentsline{toc}{subsection}{Disposició addicional única. Gènere}
Les referències a persones fetes en aquest reglament o altres documents de la Delegació s’entenen referides a persones de qualsevol identitat de gènere, excepte si s’indica explícitament el contrari.


\end{document}
