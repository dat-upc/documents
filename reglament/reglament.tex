\documentclass[a4paper,12pt]{article}

\usepackage[catalan]{babel}
\usepackage{parskip}
\usepackage{graphicx}
\usepackage[shortlabels]{enumitem}
\usepackage[hidelinks]{hyperref}
\usepackage[margin=3cm]{geometry}
\usepackage{microtype}

% Canviar la font del document
\usepackage{fontspec}
\setmainfont{Libertinus Serif}

% Evitar que es faci reset al comptador dels articles
\usepackage{chngcntr}
\counterwithout{subsubsection}{subsection}

% Canviar el prefix de l'índex
\usepackage{tocloft}
\renewcommand*{\cftsecpresnum}{Capítol }
\renewcommand*{\cftsecaftersnum}{.}
\setlength\cftsecnumwidth{5.5em}
\renewcommand*{\cftsubsecpresnum}{Secció }
\renewcommand*{\cftsubsecaftersnum}{.}
\setlength\cftsubsecnumwidth{4em}
\renewcommand*{\cftsubsubsecpresnum}{Article }
\renewcommand*{\cftsubsubsecaftersnum}{.}
\setlength\cftsubsubsecnumwidth{5em}

% Canviar el nom de les seccions, subseccions, etc.
\usepackage{titlesec}
\renewcommand*{\thesection}{\Roman{section}}
\renewcommand*{\thesubsection}{\arabic{subsection}}
\renewcommand*{\thesubsubsection}{\arabic{subsubsection}}
\titleformat{\section}[block]
{\normalfont\Large\bfseries}{Capítol~\thesection.}{1em}{}
\titleformat{\subsection}
{\normalfont\large\bfseries}{Secció~\thesubsection.}{1em}{}
\titleformat{\subsubsection}
{\normalfont\normalsize\bfseries}{Article~\thesubsubsection.}{1em}{}

% Títol del document
\title{Reglament de la Delegació d'Alumnes de Telecomunicacions}
\makeatletter

\begin{document}
% Portada
\thispagestyle{empty}
\begin{center}
\includegraphics[scale=0.8]{Dat_logo_10x6.png}
\vspace{3em}

\Huge
\@title
\vspace{1em}

\large
\today
\end{center}
\normalsize
\newpage

\thispagestyle{empty}
\tableofcontents
\newpage

\section{Disposicions generals}
\subsubsection{Naturalesa}
D'acord amb l'Article 98.1 dels Estatuts de la UPC, la Delegació d'Estudiants és l'òrgan de coordinació dels representants dels estudiants de grau i màster universitari dins l'àmbit de cada centre docent.

Aquest òrgan es defineix com la Delegació d'Alumnes de Telecomunicacions. A partir d'aquest moment es definirà i s'anomenarà DAT o Delegació en aquest document, així com en la seva identitat física i digital.

\subsubsection{Composició i mandat}
La composició de la Delegació d'estudiants és la que s'estableix a l'Article 35 del Reglament d'organització i funcionament de l'ETSETB. A més a més, formaran part de la Delegació els claustrals que són estudiants de l'Escola.

\subsubsection{Funcions}\label{art:funcions}
La Delegació d'estudiants té les funcions que li atorguen els Estatuts de la UPC, les Directrius per a l'elaboració dels Reglaments d'organització i funcionament dels centres docents, i el Reglament d'organització i funcionament de l'ETSETB.

La Delegació d'estudiants també té les següents funcions:
\begin{enumerate}[a)]
	\item Elaborar anualment les línies estratègiques i objectius de la Delegació d'estudiants.
	\item Gestionar i fer la liquidació del pressupost que té assignat la Delegació.
	\item Informar als estudiants de grau i màster sobre la tasca que duu a terme la Delegació, així com d'altres aspectes que puguin ser d'interès pels estudiants.
	\item Potenciar i canalitzar la participació dels estudiants en els òrgans de govern de la Universitat i de l'Escola.
	\item Contribuir a l'assessorament dels estudiants de l'Escola i vetllar pels seus drets.
\end{enumerate}

\section{Drets i deures}
\subsubsection{Drets dels membres de la delegació}
A més dels drets que es reconeixen al Reial Decret 1791/2010, de 30 de desembre, pel qual s'aprova l'Estatut de l'estudiant universitari, als Estatuts de la UPC i la resta de normativa que sigui d'aplicació els membres de la delegació tenen els següents drets:

\begin{enumerate}[a)]
	\item Expressar amb llibertat les manifestacions que realitzin en l'exercici de les seves funcions.
	\item Participar a les sessions dels òrgans als quals es pertanyi amb veu i vot.
	\item Ser elector i elegible per a càrrecs de gestió i representació de la Delegació d'estudiants o per formar part de les seves comissions, si escau.
	\item Accedir a les instal·lacions, materials i documents de la Delegació en els termes que s'estableixin
	\item No ser perjudicat per exercir la representació dels estudiants.
	\item Fer ús de la paraula i intervenir d'acord amb els deures de convivència.
\end{enumerate}

\subsubsection{Deures dels membres de la delegació}
A més dels deures que es detallen al Reial Decret 1791/2010, de 30 de desembre, pel qual s'aprova l'Estatut de l'estudiant universitari, als Estatuts de la UPC i la resta de normativa que sigui d'aplicació els membres de la delegació tenen els següents deures:

\begin{enumerate}[a)]
	\item Complir amb aquest Reglament i els acords que s'adoptin.
	\item Assistir a les sessions del Ple i als actes que requereixen la seva presència, així com a la resta d'òrgans als quals es pertanyi.
	\item Transmetre les opinions i peticions col·lectives dels seus representats.
	\item Informar als seus representats de les activitats realitzades en l'exercici del seu càrrec.
	\item Fer un bon ús dels espais, materials, documentació i la resta de mitjans necessaris per a l'exercici de les seves funcions o drets.
	\item Respectar les decisions dels òrgans de la Delegació i no fer pública informació confidencial de la Delegació.
\end{enumerate}

\section{Òrgans col·legiats i unipersonals}
\subsection{Disposicions generals}
\subsubsection{Òrgans col·legiats}
Els òrgans col·legiats de la Delegació d'estudiants són:
\begin{enumerate}[a)]
	\item El Ple.
	\item El Consell de la Delegació.
\end{enumerate}

\subsubsection{Òrgans unipersonals}
Els òrgans unipersonals de la Delegació són:
\begin{enumerate}[a)]
	\item El delegat de centre.
	\item El secretari.
	\item Els sotsdelegats de centre.
	\item El tresorer.
\end{enumerate}

\subsubsection{Disposicions generals sobre els òrgans col·legiats}
\begin{enumerate}[\thesubsubsection.1]
	\item L'assistència a les sessions dels òrgans col·legiats té caràcter personal i el vot és intransferible.
	\item La manca d'assistència no excusada a dues sessions consecutives de l'òrgan n'implica el cessament com a membre.
\end{enumerate}

\subsubsection{Convocatòria i sessions dels òrgans col·legiats}\label{art:convocatories}
\begin{enumerate}[\thesubsubsection.1]
	\item La convocatòria a les sessions ordinàries del Ple correspon al president o la presidenta de l'òrgan col·legiat, i es tramet amb una antelació mínima de set dies pels mitjans electrònics admesos per la Universitat, i ha d'especificar el lloc, la data, l'hora i l'ordre del dia de la reunió.
	\item El delegat de centre és el president del Ple i del Consell de la Delegació.
	\item La convocatòria ha d'especificar el lloc o l'adreça d'accés als documents que corresponen a l'ordre del dia, que s'han de donar a conèixer set dies abans de la data de la sessió. Excepcionalment, els documents que no es poden difondre amb l'antelació prevista han d'estar a disposició dels membres de l'òrgan col·legiat, en tot cas, el dia hàbil anterior al de la sessió.
	\item Els òrgans col·legiats també poden reunir-se en sessió extraordinària si hi ha temes d'urgència que ho justifiquin; en aquest cas la iniciativa de la convocatòria correspon al seu president o bé a un 20\% dels membres de l'òrgan.\label{art:extraordinaries}
	L'ordre del dia d'una reunió extraordinària es reduirà als temes que l'han motivada i es farà públic en convocatòria amb un antelació mínima de dos dies hàbils.
	\item Perquè la constitució de l'òrgan col·legiat sigui vàlida cal la presència del president i el secretari, o, si s'escau les persones que els substitueixen, i el 30\%, com a mínim, dels membres de l'òrgan col·legiat. Si no hi ha quòrum, l'òrgan col·legiat es constitueix en segona convocatòria i cal que hi assisteixin, com a mínim, un 15\% dels seus membres.\label{art:constitucio}
	\item Els òrgans col·legiats de la Delegació poden dur a terme les seves funcions de manera no presencial seguint les directrius recollides al Reglament de les sessions a distància dels òrgans col·legiats, Acord del Consell de Govern CG/2020/07/11 i ratificat per la Junta de l'ETSETB.
\end{enumerate}

\subsubsection{Adopció d'acords}\label{art:acords}
\begin{enumerate}[\thesubsubsection.1]
	\item Els acords s'adopten quan el nombre de vots favorables és superior al nombre de vots desfavorables, excepte aquells acords que requereixin majoria qualificada i estiguin explícitament definits en aquest Reglament o en altra normativa que resulti d'aplicació. En cas d'empat, el president de l'òrgan en qüestió pot emetre un vot de qualitat.
	\item La votació ha de ser secreta sempre que es tracti d'acords que afectin a persones, o bé quan ho sol·licita el president o la presidenta de l'òrgan, o bé el 20\% dels assistents a la sessió.
\end{enumerate}

\subsection{El Ple}
\subsubsection{Naturalesa i composició}
El Ple és l'òrgan superior de representació de la Delegació d'estudiants del centre. Formen part del Ple tots els membres de la Delegació.

El delegat de centre és el president del Ple.

El secretari de la Delegació actuarà com a secretari del Ple i donarà fe dels acords presos.

Al Ple hi poden assistir amb veu però sense vot:
\begin{enumerate}[a)]
	\item Qualsevol estudiant de l'ETSETB.
	\item Qualsevol altra persona convidada per un membre del Ple.
\end{enumerate}

\subsubsection{Funcions}
El Ple exerceix les següents funcions:
\begin{enumerate}[a)]
	\item Les funcions definides a l'Article \ref{art:funcions} d'aquest Reglament.
	\item Elegir i cessar el delegat de centre.
	\item Cessar els sotsdelegats de centre.
	\item Crear i dissoldre les comissions que consideri oportunes amb les finalitats i atribucions que el mateix Ple decideixi.
	\item Aprovar el Reglament de funcionament de la Delegació.
	\item Proposar iniciatives, aspiracions i manifestar la seva opinió sobre problemes que afectin la comunitat universitària.
\end{enumerate}

\subsubsection{Delegació de funcions}
El Ple pot delegar a les comissions les seves funcions. Tanmateix, no són delegables:
\begin{enumerate}[a)]
	\item L'elecció i el cessament del delegat de centre.
	\item El cessament dels sotsdelegats de centre.
	\item L'aprovació del Reglament de la Delegació.
	\item L'aprovació del Pressupost anual i el tancament de l'exercici anterior
\end{enumerate}

\subsubsection{Sessions}
\begin{enumerate}[\thesubsubsection.1]
	\item El Ple es reuneix en sessió ordinària, com a mínim, una vegada al quadrimestre.
	\item El Ple es reuneix en sessió extraordinària en els termes regulats a l'Article \ref{art:extraordinaries} d'aquest Reglament.
\end{enumerate}

\subsubsection{Funcionament}
S'apliquen l'Article \ref{art:convocatories} i l'Article \ref{art:acords} d'aquest Reglament pelr que fa a la convocatòria i adopció d'acords.

\subsection{El Consell de la Delegació}
\subsubsection{Naturalesa i funcions}
El Consell de la Delegació exerceix les següents funcions:
\begin{enumerate}[a)]
	\item Aprovar la creació i/o dissolució, quan s'escaigui, de les grups de treball que es considerin convenients.
	\item Escollir i revocar els membres dels grups de treball creats pel Consell de la Delegació.
	\item Escollir i revocar els membres de les comissions de l'ETSETB.
	\item Elaborar una proposta per a la gestió dels recursos econòmics de la Delegació, per a la seva posterior aprovació per part del Ple.
	\item Assessorar el delegat de centre en les seves funcions.
	\item Les funcions indicades a l'Article 21 del Reglament de l'ETSETB.
	\item Proporcionar la informació i assessorament necessari als estudiants que ho requereixin.
	\item Donar suport a les activitats organitzades per la Delegació.
	\item Mediar entre els membres del Ple per assegurar-ne la bona convivència.
\end{enumerate}

\subsubsection{Mandat}
El mandat del Consell de la Delegació serà el corresponent al del delegat de centre.

\subsubsection{Composició}
El Consell de la Delegació està format per:
\begin{enumerate}[a)]
	\item El delegat de centre, que n'és el president.
	\item El secretari de la Delegació, que n'és el secretari.
	\item Els sotsdelegats de centre.
	\item El tresorer.
\end{enumerate}

\subsubsection{Sessions}
\begin{enumerate}[\thesubsubsection.1]
	\item El Consell de la Delegació es reuneix amb periodicitat mensual.
	\item El Consell es reuneix en sessió extraordinària en els termes regulats a l'Article \ref{art:extraordinaries} d'aquest Reglament.
	\item Hi poden assistir, amb veu però sense vot, els estudiants de la UPC que siguin convidats pel delegat de centre.
\end{enumerate}

\subsubsection{Funcionament}
\begin{enumerate}[\thesubsubsection.1]
	\item Pel funcionament del Consell de la Delegació s'apliquen l'Article \ref{art:constitucio} i l'Article \ref{art:acords} d'aquest Reglament.
	\item No es podrà prendre un acord que correspongui al delegat de centre, al secretari o al tresorer si aquests no han pogut assistir a la sessió.
\end{enumerate}

\subsection{El delegat de centre}
\subsubsection{Naturalesa}
D'acord amb l'Article 99 dels Estatuts de la UPC el delegat de centre de l'Escola exerceix la funció de portaveu de la Delegació i la màxima representació dels estudiants del seu centre.

\subsubsection{Funcions}
El delegat de centre té les funcions que li assignen els Estatuts de la UPC, i el Reglament d'organització i funcionament de l'Escola.

El delegat de centre també té les següents funcions:
\begin{enumerate}[a)]
	\item Vetllar pel compliment d'aquest Reglament i el bon funcionament de la Delegació.
	\item Nomenar i destituir els sotsdelegats, el secretari i el tresorer.
	\item Convocar i presidir les reunions del Ple i del Consell de la Delegació.
	\item Convocar eleccions a delegat de centre.
\end{enumerate}

\subsubsection{Elecció}
El Ple de la delegació elegeix el delegat de centre entre els membres que formen part del Ple.

Per a l'elecció del delegat de centre s’aplica el procediment previst per a l'elecció de director de departament, regulat al Capítol 5 del Títol IV del Reglament electoral de la UPC.

\subsubsection{Mandat}
El mandat és el que designa el Reglament d’organització i funcionament de l'ETSETB en l'Article 37.

\subsubsection{Cessament}
\begin{enumerate}[\thesubsubsection.1]
	\item El Ple de la delegació pot cessar el delegat de centre.
	\item La proposta de cessament l'ha de presentar, com a mínim, un terç dels membres del Ple. La proposta ha d'incloure el calendari electoral d'aplicació.
	\item L'aprovació del cessament requereix el vot favorable de dos terços dels membres del Ple.
	\item Si la proposta de cessament no s'aprova, els signataris no en poden presentar una altra fins al cap d'un any.
	\item Si el Ple n'aprova el cessament, el Ple convoca eleccions de forma immediata i el delegat de centre cessa en les seves funcions.
\end{enumerate}


\subsection{Els sotdelegats de centre}
\subsubsection{Nomenament i cessament}
\subsubsection{Funcions}

\subsection{El secretari}
\subsubsection{Naturalesa i funcions}

\subsection{El tresorer}
\subsubsection{Naturalesa i funcions}

\section{Aprovació i modificació del Reglament de la Delegació}
\subsubsection{Titulars de la iniciativa}
Poden promoure la proposta de modificació d'aquest reglament:
\begin{enumerate}[a)]
	\item El delegat de centre de l'ETSETB.
	\item Un mínim del 30\% dels membres del Ple.
\end{enumerate}

\subsubsection{Procediment}
\begin{enumerate}[\thesubsubsection.1]
	\item La proposta de modificació del reglament de funcionament ha d'anar acompanyada d'un escrit de motivació i ha d'incloure el text de modificació proposat.
	\item L'aprovació de la proposta de modificació d'aquest reglament requereix un quòrum de participació de dues terceres parts del nombre d'assistents al Ple on s'ha proposat la modificació i el vot favorable, com a mínim, de la meitat més un dels vots vàlids emesos.
	\item Un cop aprovada la proposta, l'expedient es tramet a la Junta del centre docent perquè l'aprovi, si escau.
\end{enumerate}

\section*{Disposicions addicionals}
\addcontentsline{toc}{section}{Disposicions addicionals}

\subsubsection*{Disposició addicional única. Gènere}
\addcontentsline{toc}{subsection}{Disposició addicional única. Gènere}
Les referències a persones fetes en aquest reglament o altres documents de la Delegació s'entenen referides a persones de qualsevol identitat de gènere, excepte si s'indica explícitament el contrari.


\end{document}
